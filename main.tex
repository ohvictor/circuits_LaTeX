\input{preamble.tex}

\begin{document}

%\chapter{Amplificador No Inversor}

\begin{figure}[ht]
	\begin{center}
		\begin{circuitikz}[american voltages]
\draw
(0,0) node[op amp, noinv input up] (opamp) {}
(opamp.+) to ++(-2,0) to [sinusoidal voltage source=$V_{in}$] ++(0,-3) node [ground] {}
(opamp.out) to ++(1,0) coordinate(vo) to ++(1,0) node[right] {$v_{out}$}
(opamp.-) to ++(0,-1) coordinate(tmp)
(tmp) to[R=$R_1$,*-] ++(0,-2) node[ground] {}
(tmp) to[R=$R_2$] ++(3.25,0) to [short,-*] ++(0,1.5)
;
\end{circuitikz}
\caption{Amplificador no inversor}
	\caption{Circuito Amplificador No Inversor}
	\label{fig:ninv}
	\end{center}
\end{figure}

\begin{equation}
A = \frac{v_o}{v_i} = \left(1+\frac{R_2}{R_1}\right) \cdot \frac{1}{1+(1+R_2/R_1)/a_{VOL}}
\end{equation}

\begin{equation}
R_i=\infty \qquad R_o=0
\end{equation}
%\newpage
%\chapter{Amplificador Buffer}

\begin{figure}[ht]
	\begin{center}
		\input{buffer.tex}
	\caption{Circuito Amplificador No Inversor}
	\label{fig:buffer}
	\end{center}
\end{figure}

\begin{equation}
A = 1 V/V \qquad R_i=\infty \qquad R_o=0
\end{equation}
%\newpage
%\chapter{Amplificador Inversor}

\begin{figure}[ht]
	\begin{center}
		\begin{circuitikz}[american voltages]
\draw
(0,0) node[op amp] (opamp) {}
(opamp.out) to ++(2,0) node[right] {$v_{out}$}
(opamp.-) to ++(0,1) coordinate(tmp)
(tmp) to[R=$R_1$,*-] ++(-3,0) to [sinusoidal voltage source=$V_{in}$] ++(0,-2) node [ground] {}
(tmp) to[R=$R_2$] ++(3,0) to [short,-*] ++(0,-1.5)
(opamp.+) to ++(0,-1) node[ground]{}
;
\end{circuitikz}
	\caption{Circuito Amplificador No Inversor}
	\label{fig:inv}
	\end{center}
\end{figure}

\begin{equation}
A = \frac{v_o}{v_i} = \left(-\frac{R_2}{R_1}\right) \cdot \frac{1}{1+(1+R_2/R_1)/a_{VOL}}
\end{equation}

\begin{equation}
R_i=R_1 \qquad R_o=0
\end{equation}
%\newpage


\begin{figure}
    \begin{center}
        \input{TC/rauch.tex}
    \end{center}
\end{figure}

\begin{figure}
    \begin{center}
        \begin{circuitikz}[american voltages]
\draw
(0,0) node[op amp, noinv input up] (opamp) {}
(opamp.+) to ++(-2,0) to [sinusoidal voltage source=$V_{in}$] ++(0,-3) node [ground] {}
(opamp.out) to ++(1,0) coordinate(vo) to ++(1,0) node[right] {$v_{out}$}
(opamp.-) to ++(0,-1) coordinate(tmp)
(tmp) to[R=$R_1$,*-] ++(0,-2) node[ground] {}
(tmp) to[R=$R_2$] ++(3.25,0) to [short,-*] ++(0,1.5)
;
\end{circuitikz}
\caption{Amplificador no inversor}
    \end{center}
\end{figure}

\begin{figure}
    \begin{center}
        \input{TC/sallen-key.tex}   
    \end{center}
\end{figure}

\begin{figure}
    \begin{center}
        \begin{circuitikz}
            \draw
            (0,0) node[npn,xscale=-1](T2){\ctikzflipx{$T_2$}}
            (T2.E) ++(0,-2) node[sground](g0){} to[zzDo,l=$D_z$] (T2.E)
            (T2) ++(0,3) node[npn,rotate=90](T1){\rotatebox{-90}{$T_1$}}
            (T1.E) to[short,o-o] ++(2,0) coordinate(x1)
            (g0) ++(3,0) coordinate(g1)
            (g1) to[R,l=$R_5$] ++(0,2) to[pR,l=$R_4$, wiper pos = 0.3,n=R4] ++(0,2) to[R,l=$R_3$] (x1)
            (T2.B) to[short] (R4.wiper)
            (g1) to[short] ++(2,0) to[vR, l=$R_L$] ++(0,5.5) to[short](x1)
            ;
        \end{circuitikz}
        \caption{Regulador Serie que utiliza un lazo de realimentación negativa que muestrea tensión y suma tensión}
    \end{center}
\end{figure}

\end{document}